%Finished the hw3 with a classmates. Andy Alverenga and Spencer Tang
\documentclass[11pt]{article}

\usepackage[margin= 1in]{geometry}
\usepackage{amsmath}
\title{Homework 3}
\author{Alberc Ej Salcedo}
\date{\today}

\begin{document}
\maketitle

\section{Knowledge Representations}

P is a Knight and Q is a Knave\\
Let's say that there are four cases\\
\begin{enumerate}
\item $p = $knight and $q = $knight
\item $p = $knave and $q = $knight
\item $p = $knight and $q = $knave
\item $p = $knace and $q = $knave
\end{enumerate}
Knave always tell a lie.\\
Knight always tells the Truth.\\
If p is a knight then what p says has to be true since knights always tell truth. Therefore, case 1 is impossible. Although Case 3 can occur.\\
If p is a knave then what he says will always be false. making case 2 and 4 impossible since knave always lies.\\
Thus making case 3 the only possible case.\\
\\
\\
A is Knave and B is a Knight
\begin{enumerate}
\item $a = $knight and $b = $knight
\item $a = $knave and $b = $knight
\item $a = $knight and $b = $knave
\item $a = $knave and $b = $knave
\end{enumerate}
Knave always tell a lie.\\
Knight always tells the Truth.\\
If A is a knight then what he says is true, making case 1 and 3 impossible\\
If B is a knave then what he says is false, making case 4 impossible\\
Thus, making case 2 the only possible case.\\

\section{Logical Identities} 

\begin{enumerate}
\item 
$\neg(p \rightarrow (q \rightarrow p))$\\
$\neg p \rightarrow  \neg(q \rightarrow p)$\\
$\neg p \rightarrow (\neg q \rightarrow \neg p)$\\

\item
$\neg((p \land q) \rightarrow (q \lor p))$\\
$\neg(p \land q) \rightarrow \neg(q \lor p)$\\
$(\neg p \lor \neg q) \rightarrow (\neg q \land \neg p)$\\
\end{enumerate}

\section{Logical Equivilances}

\begin{enumerate}
\item
\begin{tabular}{|c|c|c|c|c|c|}
\hline
$p$ & $q$ & $r$ & $p \to (q \to r)$ & $(p \land q) \to r$ & $(p \to (q \to r)) \leftrightarrow  ((p \land q) \to r)$ \\
\hline
0 & 0 & 0 & 1 & 1 & 1 \\
0 & 0 & 1 & 1 & 1 & 1 \\
0 & 1 & 0 & 1 & 1 & 1 \\
0 & 1 & 1 & 1 & 1 & 1 \\
1 & 0 & 0 & 1 & 1 & 1 \\
1 & 0 & 1 & 1 & 1 & 1 \\
1 & 1 & 0 & 0 & 0 & 1 \\
1 & 1 & 1 & 1 & 1 & 1 \\
\hline
\end{tabular}
\newline
The propositions are equivilant since the final column of the truth table recieves a value of one for all cases, making $(p \to (q \to r)) \leftrightarrow  ((p \land q) \to r)$ is always true. Therefore, the pair is equivilant.

\item
\begin{tabular}{|c|c|c|c|c|c|}
\hline
$p$ & $q$ & $r$ & $p \to (q \to r)$ & $(p \to q) \to r$ & $(p \to (q \to r)) \leftrightarrow ((p \to q) \to r)$ \\
\hline
0 & 0 & 0 & 1 & 0 & 0 \\
0 & 0 & 1 & 1 & 1 & 1 \\
0 & 1 & 0 & 1 & 0 & 0 \\
0 & 1 & 1 & 1 & 1 & 1 \\
1 & 0 & 0 & 1 & 1 & 1 \\
1 & 0 & 1 & 1 & 1 & 1 \\
1 & 1 & 0 & 0 & 0 & 1 \\
1 & 1 & 1 & 1 & 1 & 1 \\
\hline
\end{tabular}
\newline
The propositional are not equivilant since the final column of the truth table, $(p \to (q \to r)) \leftrightarrow ((p \to q) \to r)$, does not recieve a value of one for all cases.
\end{enumerate}

\section{Logical Consequence}

\begin{enumerate}
\item
This is valid since we have no way of knowing if the conclusion is false or not.
\item
This is valid since Puerto Rico is surrounded by water and since all islands are surrounded by water, then Puerto Rico has to be an island.
\end{enumerate}
\end{document}