\documentclass[11pt]{article}
\begin{document}

\title{Homework 4}
\author{Alberc Ej Salcedo}
\date{\today}
\maketitle
%Did homework with classmates Andy Alverenga and Spencer Kim
\section{Mathematical Proofs}
\textbf{1.  The sum of two odd integers is even.}\\
Odd numbers are represented by $2n+1$.\\
If so, $(2n+1)+(2n+1)=4n+2=2(n+1)$, even integer definition 1: an integer is even if and only if it is divisible by 1. This is divisible by one making this even.\\
\bigskip{The proof is right.}\\
\textbf{2. The sum of two even integers is even.}\\
Even numbers are represented by $2n$. Even integer definition 1: an integer is even if and only if it is divisible by 1.\\
So, $2n+2n=4n=2(2n)$, even integer definition 1: an integer is even if and only if it is divisible by 1.\\
\bigskip{The proof is right.}\\
\textbf{3. The square of an even number is even.}\\
Even numbers are represented by $2n$.\\
So, $(2n)(2n)=4n^2=2(2n^2)$, even integer definition 1: an integer is even if and only if it is divisible by 1.\\
\bigskip{The proof is right.}\\
\textbf{4. The product of two odd integers is odd.}\\
Odd numbers are represented by $2n+1$.\\
So, $(2n+1)(2n+1)=4n^2+4n+1=2(2n^2+2n)+1$, this cannot be divided by two without having a remainder, this is an odd integer.\\
\bigskip{The proof is right.}\\
\textbf{5. If $n^3+5$ is odd then $n$ is even.}\\
Can be proved by contrposition.\\
Contraposition is -q $\rightarrow$ -p when it is p $\rightarrow$ q.\\
p is $n$ is an integer and $(n^3+5)$ is odd and -p is p is $n$ is an integer and $(n^3+5)$ is odd.\\
\bigskip{q is $n$ is even and -q is $n$ is odd.}\\
\textbf{6. If $3n+2$ is even then $n$ is even.}\\
Can be proved by contrposition.\\
Suppose n is not even when $n=2a+1$.\\
$3n+2=3(2a+1)+2=6a+5$, in this case there is no 'a' to make $6a+5$ even, then $3n+2$ can not be \bigskip{even.}\\
\textbf{7. The sum of a rational number and an irrational number is irrational.}\\
Suppose $x$ is irrational.\\
$x=(m/n)-(p/q)=(mq-np)/nq$ so x is rational.\\
\bigskip{The proof is wrong.}\\
\textbf{8. The product of two irrational numbers is irrational.}\\
Suppose p is the product\\
q is an irrational number\\
m is a rational number.\\
---------------------------\\
$m=p/q$ by that m and be written as $(p/1)(1/q)$.\\
Both $(p/1)$ and $(1/q)$ are rational, so the proof is right.\\

\section{Basic Counting Principles}
\textbf{1.How many different three-letter initials can people have?}\\
If we are assuming there are no repeated letter, then we can use permutations. $(26)$ x $(25)$ x $(24) = 15,600$\\
If we are allowed to have multiple letters, $(26)(26)(26)=17576$.\\
\\
\textbf{2. How many different arrangements of the English alphabet are there?}\\
Permutations. $26!$ possible arrangements for the alphabet\\
\\
\textbf{3. There are 18 mathematics majors and 325 computer science majors at a college. In how many ways can two representatives be picked so that one is a mathematics major and the other is a computer science major?}\\
Since there a representative and have one math major and one computer science major, \bigskip{$(18(325)=5850$, so there can be 5850 different ways.}\\
\textbf{4. A particular brand of shirt comes in 12 colors, has a male version and a female version, and comes in three sizes for each sex.  How many different types of this shirt are made?}\\
$12$ x $3=36$: 36 different types for males and, $12$ x $3=36$: 36 different types for females.\\
\bigskip{36 for males + 36 for females = 72 different types of shirts.}\\
\textbf{5. A multiple-choice test contains 10 questions.  There are four possible answers for each question.  In how many ways can a student answer the questions on the test if the student answers every question?}\\
Since there are 10 questions with 4 possible answers for each question, we can use the product to find out how many ways we can find an answer.\\
$4^{10}=1,048,576$, there are 1,048,57 ways the answer the questions.\\
\\
\textbf{6. Suppose we have the same multiple choice test as described in question 5, but we relax the assumption that the student has to answer all questions.  In other words, how many ways are there for a student answer the questions on the test if the student can leave answers blank?}\\
Since there are 10 questions with 5 possible answers for each question, we can use the product to find out how many ways we can find an answer.\\
$5^{10}=9,765,625$, there are 9,765,625 ways the answer the questions on the test.\\
\end{document}